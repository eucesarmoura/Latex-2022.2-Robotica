\documentclass[a4paper, 12pt]{article}

\usepackage[top= 0.5in]{geometry}
\usepackage[utf8]{inputenc}
\usepackage[portuguese]{babel}
\usepackage{indentfirst}
\usepackage{graphicx}

\usepackage{biblatex}
\addbibresource{referencias.bib}

\title{IF668 - Robótica}
\author{Cesar Henrique Correia de Moura}
\date{Setembro, 2022}

\begin{document}

\maketitle


\section{Introdução}
\par
Robótica é uma disciplina eletiva cuja a função principal é apresentar conceitos gerais do estudo das tecnologias associadas a concepção e construção de robôs, bem como algumas diferenças entre eles e aplicações dos mesmos no contexto tecnológico atual.  No geral, contibui com uma base para o entendimento e aprofundamento do tema para estudantes que tenham interesse em atuar de alguma forma na área.\cite{sitecinIC}
\par
\begin{figure}[h]
    \centering
    \includegraphics[width=1\textwidth]{Imagens/evolucao.jpg}
    \caption{Representação da evolução de um robô móvel ao longo do tempo\cite{artigo_computadores}}
    \label{fig:evolucao}
\end{figure}
    
    
\section{Relevância}
    \par
A disciplina é de vital importância em virtude de ser a síntese de muitos conhecimentos vistos em cadeiras como Engenharia de Sistemas Embutidos (IF728) e Controle e Manipulação (IF835). Tal cadeira busca racionalizar e explicar de forma objetiva e clara a atuação de ferramentas como sensores e robôs movéis, bem como a forma com as quais esses podem ser programados. Além disso, por meio de projetos e trabalhos os alunos também podem por em prática os conhecimentos técnicos do desenvolvimento e manipulação de robôs.\cite{sitecinPERFIL}


\section{Relação com outras disciplinas}
    \par
    \begin{itemize}
        \item IF728 - ENGENHARIA DE SISTEMAS EMBUTIDOS
            \par 
             Sistema embarcado (ou sistema embutido) é um sistema microprocessado no qual o computador é completamente encapsulado ou dedicado ao dispositivo ou sistema que ele controla. Ao longo dessa disciplina serão trabalhados conteúdos essenciais a atuação com robôs na disciplina de robótica. \cite{sitecinHFC}
        \item IF669 - CONTROLE E MANIPULAÇÃO
            \par
            Os principais tópicos dessa cadeira são a utilização de sensores, linguagens e ambientes de programação para o controle de robôs. Junto com Engenharia de Sistemas embutidos, ambas as disciplinas contemblam os conteúdos básicos para projetar e programar funções em robôs.\cite{sitecinHFC}
        \item IF679 - INFORMÁTICA E SOCIEDADE
            \par 
            A disciplina de Informática e Sociedade visa trabalhar vários tópicos como questões de esfera social e filósofica ligados a tecnologia. Dentre eles está a questão ética que envolve a utilização de robôs na contemporaneidade que torna-se crescente e abre espaço para diversas novas questões e problemáticas ligadas a nossa nova convivência com as máquinas. \cite{sitecinHFC}
    \end{itemize}

\printbibliography

\end{document}
